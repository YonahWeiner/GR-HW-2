\documentclass{article}
\usepackage{graphicx} % Required for inserting images
\usepackage{physics}
\usepackage{amsmath}

\title{GR HW 2}
\author{Yonah Weiner 337756787}
\date{June 18, 2024}

\begin{document}

\maketitle

\subsection*{Question 1}
\subsubsection*{1. Show that the entire path is contained within a sphere.}
\subsubsection*{Solution}
The trajectory in spherical coordinates is: 
\begin{equation}
        \gamma'^i(\lambda) =  \begin{pmatrix}
            R\\
            \psi\\
            \omega \lambda
        \end{pmatrix}
\end{equation}
We see that $R$ is constant for any $\lambda$, therefore the entire trajectory is contained within a sphere radius $R$.
\subsubsection*{2. Find the velocity vector in the given coordinates}
\subsubsection*{Solution}
We calculate the velocity in the Cartesian coordinates:
\begin{equation}
        \dv{\lambda}\gamma^i(\lambda) = \dv{\lambda}( R\begin{pmatrix}
            \sin \psi \cos (\omega\lambda)\\
            \sin \psi \sin (\omega\lambda)\\
            \cos \psi 
        \end{pmatrix})= R\omega\begin{pmatrix}
           -\sin \psi \sin (\omega\lambda)\\
           \sin \psi \cos (\omega\lambda)\\
            0
        \end{pmatrix}
\end{equation}

\subsubsection*{3. Calculate the velocity vector in spherical coordinates by two methods:}
\paragraph{(a) Taking a derivative of $\lambda'^i$}
\paragraph{Solution}

\begin{equation*}
    \dv{\lambda}\gamma'^i(\lambda) = \dv{\lambda} \begin{pmatrix}
            R\\
            \psi\\
            \omega \lambda
        \end{pmatrix}= \begin{pmatrix}
           0\\
           0\\
            \omega
        \end{pmatrix}
\end{equation*}
\paragraph{(b) Applying the transformation law on the the velocity vector in the original coordinates}
The transformation matrix is given by:
\begin{equation*}
\begin{split}
    \dv{x'}{x} &=\begin{pmatrix}
        \dv{r}{x} & \dv{r}{y} & \dv{r}{z}\\
        \dv{\psi}{x} & \dv{\psi}{y} & \dv{\psi}{z}\\
        \dv{\phi}{x} & \dv{\phi}{y} & \dv{\phi}{z}
    \end{pmatrix}\\
    &=\begin{pmatrix}
        \dv{x}{r} & \dv{x}{\psi} & \dv{x}{\phi}\\
        \dv{y}{r} & \dv{y}{\psi} & \dv{y}{\phi}\\
        \dv{z}{r} & \dv{z}{\psi} & \dv{z}{\phi}
    \end{pmatrix}^{-1}\\
    &=\begin{pmatrix}
        \sin \psi \cos \phi & r\cos \psi \cos \phi & -r\sin \psi \sin \phi\\
        \sin \psi \sin \phi & r\cos \psi \sin \phi & r\sin \psi \cos \phi\\
        \cos \psi  & -r\sin \psi & 0
    \end{pmatrix}^{-1}\\
    &=\text{mathematica}\\
    &=\left(
\begin{array}{ccc}
 \sin (\psi) \cos (\phi) & \sin (\psi) \sin (\phi) & \cos (\psi) \\
 \frac{\cos (\psi) \cos (\phi)}{r} & \frac{\cos (\psi) \sin (\phi)}{r} & -\frac{\sin (\psi)}{r} \\
 -\frac{\csc (\psi) \sin (\phi)}{r} & \frac{\csc (\psi) \cos (\phi)}{r} & 0 \\
\end{array}
\right)\\
&=\left(
\begin{array}{ccc}
 \sin (\psi) \cos (\omega \lambda) & \sin (\psi) \sin (\omega \lambda) & \cos (\psi) \\
 \frac{\cos (\psi) \cos (\omega \lambda)}{R} & \frac{\cos (\psi) \sin (\omega \lambda)}{R} & -\frac{\sin (\psi)}{R} \\
 -\frac{\csc (\psi) \sin (\omega \lambda)}{R} & \frac{\csc (\psi) \cos (\omega \lambda)}{R} & 0 \\
\end{array}
\right)
    \end{split}
\end{equation*}
Now we use the transformation law:
\begin{equation*}
\begin{split}
   \dv{\lambda}\gamma'^i(\lambda)  &= \left(
\begin{array}{ccc}
 \sin (\psi) \cos (\omega \lambda) & \sin (\psi) \sin (\omega \lambda) & \cos (\psi) \\
 \frac{\cos (\psi) \cos (\omega \lambda)}{R} & \frac{\cos (\psi) \sin (\omega \lambda)}{R} & -\frac{\sin (\psi)}{r} \\
 -\frac{\csc (\psi) \sin (\omega \lambda)}{R} & \frac{\csc (\psi) \cos (\omega \lambda)}{R} & 0 \\
\end{array}
\right) \begin{pmatrix}
           - R\omega\sin \psi \sin (\omega\lambda)\\
            R\omega\sin \psi \cos (\omega\lambda)\\
            0
        \end{pmatrix} = \left(\begin{array}{c}
             0 \\
             0 \\
             \omega
        \end{array}\right)
\end{split}
\end{equation*}
\subsection*{Question 2}
In the tutorial we have discussed integral lines of a given vector field. find the integral lines of the following vector fields:

\end{document}
